\documentclass[10pt]{extarticle}
\usepackage{amsmath}
\usepackage{amssymb}
\usepackage{geometry}
\usepackage{multicol}
\usepackage{enumitem}
\geometry{margin=0.5in, top=0.4in, bottom=0.4in}
\setlength{\parindent}{0pt}
\setlength{\parskip}{3pt}
\setlist{nosep, leftmargin=10pt}
\setlength{\columnsep}{15pt}

\title{\vspace{-1.2cm}\LARGE\textbf{Statistics Quick Reference}}
\author{}
\date{}

\begin{document}
\maketitle
\vspace{-0.6cm}
\begin{multicol}{2}
\normalsize

\textbf{1. Sample Mean:} $\bar{x} = \frac{\sum x_i}{n} = \frac{x_1 + x_2 + \cdots + x_n}{n}$

\textit{What it does:} Calculates the average value of a dataset by summing all values and dividing by count. 

\textit{Use:} Measures central tendency (center point of data). 

\textit{Variables:} $\bar{x}$ = sample mean (average), $x_i$ = individual data values, $n$ = total number of observations, $\sum$ = sum all values.

\vspace{3pt}

\textbf{2. Sample Variance:} $s^2 = \frac{\sum (x_i - \bar{x})^2}{n - 1}$

\textit{What it does:} Measures how spread out data is from the mean by averaging squared deviations. 

\textit{Use:} Quantifies variability/dispersion in dataset. 

\textit{Variables:} $s^2$ = variance, $x_i$ = each data value, $\bar{x}$ = sample mean, $n$ = sample size, $(n-1)$ = degrees of freedom (corrects bias for sample vs. population).

\vspace{3pt}

\textbf{3. Sample Standard Deviation:} $s = \sqrt{\frac{\sum (x_i - \bar{x})^2}{n - 1}} = \sqrt{s^2}$

\textit{What it does:} Square root of variance to return to original units. 

\textit{Use:} Shows typical spread/deviation from mean in data's original units. 

\textit{Variables:} $s$ = standard deviation, $s^2$ = variance.

\vspace{3pt}

\textbf{4. Empirical Rule (68-95-99.7):} 68\% within $\bar{x}\pm s$, 95\% within $\bar{x}\pm 2s$, 99.7\% within $\bar{x}\pm 3s$.

\textit{What it does:} Approximates data distribution for normal data. 

\textit{Use:} Quick estimation of where data falls. 

\textit{Variables:} $\bar{x}$ = mean, $s$ = std dev.

\vspace{3pt}

\textbf{5. Complement Rule:} $P(A') = 1 - P(A)$

\textit{What it does:} Finds probability that event A does NOT occur. 

\textit{Use:} When calculating "not A" is easier than finding "A" directly. 

\textit{Variables:} $P(A')$ = probability of not A (complement), $P(A)$ = probability of event A occurring.

\vspace{3pt}

\textbf{6. Addition Rule:} $P(A \cup B) = P(A) + P(B) - P(A \cap B)$

\textit{What it does:} Finds probability of either event occurring. 

\textit{Use:} Combines probabilities, avoids double-counting overlap. 

\textit{Variables:} $P(A \cup B)$ = prob. A or B, $P(A \cap B)$ = prob. both A and B.

\vspace{3pt}

\textbf{6a. Independent Events:} $P(A \cap B) = P(A) \cdot P(B)$

\textit{What it does:} Calculates probability of both events occurring when they're independent. 

\textit{Use:} Test for independence: if this equation holds, events are independent; outcome of one doesn't affect the other. 

\textit{Variables:} $P(A \cap B)$ = prob. both A and B occur, $P(A)$ = prob. of A, $P(B)$ = prob. of B.

\vspace{3pt}

\textbf{6b. Mutually Exclusive (Disjoint) Events:} $P(A \cap B) = 0$

\textit{What it does:} States that both events cannot occur simultaneously. 

\textit{Use:} Test for mutual exclusivity: if events can't happen together, their intersection is zero. When disjoint: $P(A \cup B) = P(A) + P(B)$. 

\textit{Variables:} $P(A \cap B)$ = prob. both occur (zero for disjoint events).

\vspace{3pt}

\textbf{7. Expected Value:} $E(X) = \mu = \sum x \cdot p(x) = x_1p(x_1) + x_2p(x_2) + \cdots + x_np(x_n)$

\textit{What it does:} Calculates theoretical mean/average of a probability distribution by weighting each value by its probability. 

\textit{Use:} Predicts long-run average value over many trials. 

\textit{Variables:} $E(X)$ = expected value, $\mu$ = population mean, $x$ = possible outcome values, $p(x)$ = probability of each outcome.

\vspace{3pt}

\textbf{8. Variance of Random Variable:} $V(X) = \sigma^2 = \sum (x - E(X))^2 \cdot p(x)$

\textit{What it does:} Measures spread of probability distribution by averaging squared deviations from expected value. 

\textit{Use:} Quantifies uncertainty/variability in random variable. 

\textit{Variables:} $V(X)$ or $\sigma^2$ = population variance, $x$ = outcome values, $E(X)$ = expected value/mean, $p(x)$ = probabilities.

\vspace{3pt}

\textbf{9. Standard Deviation of RV:} $SD(X) = \sigma = \sqrt{V(X)}$ where $(SD_X)^2 = V(X)$

\textit{What it does:} Square root of variance. 

\textit{Use:} Shows spread in original units. 

\textit{Variables:} $SD(X)$ or $\sigma$ = population standard deviation, $V(X)$ or $\sigma^2$ = variance.

\vspace{3pt}

\textbf{10. Central Limit Theorem:} $\bar{X} \sim N(\mu, \frac{\sigma}{\sqrt{n}})$ as $n \to \infty$

\textit{What it does:} States that sample means form a normal distribution regardless of original population shape. 

\textit{Use:} Justifies using normal distribution for statistical inference with large samples. 

\textit{Variables:} $\bar{X}$ = sample mean distribution, $\mu$ = population mean, $\sigma$ = population std dev, $n$ = sample size, $\frac{\sigma}{\sqrt{n}}$ = standard error (spread of sample means).

\vspace{3pt}

\textbf{10a. Sampling Distribution of a Proportion:} $\hat{p} \sim N\left(p, \sqrt{\frac{pq}{n}}\right)$

\textit{What it does:} Describes distribution of sample proportions. 

\textit{Use:} Finding probabilities about proportions in surveys/sampling. Mean = $p$, Standard Error = $\sqrt{\frac{pq}{n}}$. 

\textit{Variables:} $\hat{p}$ = sample proportion, $p$ = population proportion, $q = 1-p$, $n$ = sample size.

\vspace{3pt}

\textbf{10b. Sampling Distribution of the Sum:} $\text{Sum} \sim N(n\mu, \sqrt{n}\sigma)$

\textit{What it does:} Describes distribution of total/sum of $n$ independent observations. 

\textit{Use:} Finding probabilities about totals rather than averages. Mean of sum = $n\mu$, Std dev of sum = $\sqrt{n}\sigma$. 

\textit{Variables:} Sum = total of $n$ values, $n$ = number of observations, $\mu$ = population mean, $\sigma$ = population std dev.

\vspace{3pt}

\textbf{11. Binomial Distribution:} $X \sim \text{Bin}(n, p)$ where $P(X=x) = \binom{n}{x}p^xq^{n-x}$, $q = 1-p$

$\binom{n}{x} = \frac{n!}{x!(n-x)!}$ \quad Moments: $E(X)=np$, $V(X)=npq$, $SD(X)=\sqrt{npq}$

\textit{What it does:} Calculates probability of exactly $x$ successes in $n$ independent trials with constant success probability. 

\textit{Use:} Fixed number of trials, two outcomes (success/failure), constant probability, independent trials. 

\textit{Variables:} $n$ = number of trials, $x$ = number of successes, $p$ = success probability, $q = 1-p$ = failure probability, $\binom{n}{x}$ = "n choose x" combinations.

\vspace{3pt}

\textbf{12. Poisson Distribution:} $X \sim \text{Pois}(\mu)$ where $P(X=x) = \frac{\mu^x e^{-\mu}}{x!}$

Moments: $E(X)=\mu$, $V(X)=\mu$, $SD(X)=\sqrt{\mu}$

\textit{What it does:} Calculates probability of exactly $x$ events occurring in a fixed interval when events happen at a known average rate. 

\textit{Use:} Rare events, known average rate, events independent. 

\textit{Variables:} $\mu$ = average rate/expected count per interval, $x$ = number of occurrences, $e$ = Euler's number (≈2.718), $x!$ = factorial of x.

\vspace{3pt}

\textbf{13. Normal Distribution:} $X \sim N(\mu, \sigma)$ with Z-score: $z = \frac{x-\mu}{\sigma}$

\textit{What it does:} Standardizes any normal variable to standard normal distribution. 

\textit{Use:} Converts to z-scores to use standard normal tables for probability calculations. 

\textit{Variables:} $X$ = original normal variable, $\mu$ = mean of distribution, $\sigma$ = standard deviation, $z$ = standard score (number of std devs from mean), Result: $Z \sim N(0,1)$.

\vspace{5pt}
\textbf{Sampling Methods:}

\vspace{2pt}
\textbf{Random Sampling:} 

\textit{What:} Every member has equal selection chance. 

\vspace{2pt}
\textbf{Stratified Sampling:} 

\textit{What:} Divide population into homogeneous groups (strata), random sample from each stratum. 

\vspace{2pt}
\textbf{Cluster Sampling (1-stage):} 

\textit{What:} Divide into clusters, randomly select some clusters, survey ALL members in selected clusters. 

\vspace{2pt}
\textbf{Cluster Sampling (2-stage):} 

\textit{What:} Divide into clusters, randomly select clusters, then random sample within selected clusters. 

\end{multicol}
\end{document}
